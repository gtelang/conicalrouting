\vspace{5mm}
\begin{description}
  \item[NAME] {\tt ls} - list directory contents
  \item[SYNOPSIS]  List  information  about the FILEs (the current directory by default).  
                   Mandatory arguments to long options are mandatory for short options too.
  \item[DESCRIPTION] \kant[1]
  \item[OPTIONS]
     \begin{description}
        \item[\opt{-a, --all}]
              do not ignore entries starting with .
        \item[\opt {-A, --almost-all}]
              do not list implied . and ..
       \item[\opt{ --author}]
              with {\tt -l}, print the author of each file
       \item[\opt{ -h, --human-readable}]
              with \opt{-l} and/or \opt{-s}, print human readable sizes 
     \end{description}
  \item[EXIT STATUS]
     \begin{itemize}
       \item[\tt{0}] if OK
       \item[\tt{1}] if minor problems (e.g. cannot access subdirectory)
       \item[\tt{2}] if serious trouble (e.g. cannot access command-line argument)
     \end{itemize}    
  \item [QUICK INSTALLATION INSTRUCTIONS] Mention what is required for the code to run, what libraries                                                                                                  
        and links to download sites. The list of dependencies can be extracted                                                                                        
        by using AWK on the files, alternatively, keep a running tag of                                                                                               
        libraries used, by adding to a separate Index which remembers what                                                                                            
        software was used. For C\texttt{++} programs, all you would have to do                                                                                        
        once the dependencies are installed to type make. Alternatively                                                                                               
        you can also provide a configur script for a larger program.                                                                                                  
        Even better just give them your docker container and tell them                                                                                                
        how to execute the program via the docker container. The docker                                                                                                
        container should be minimal, skeletal actually so that it does                                                                                                
        not take up disk space. Thus this must contain short installation                                                                                              
        instructions.   
\end{description}
